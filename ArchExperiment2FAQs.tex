\documentclass{article}
\usepackage{ctex}
\usepackage[hmargin=1.25in, vmargin=1in]{geometry}

\usepackage{fontspec}
\setmainfont{等线}
\usepackage{xeCJK}
\setCJKmainfont{等线}


\title{\textbf{体系实验二FAQs}\vspace{-2em}}
\author{}
\date{}
\begin{document}

\maketitle

\textbf{Q1:本实验中在异常/中断发生时要求改的CSR有哪些?}

A:要求的有mstatus、mcause、mepc,即在exp\_test.s中有读出来的这几个,其它自便。\\



\textbf{Q2:关于mtvec}

A:mtvec是异常/中断时处理程序的地址,启用向量模式下会对不同的异常/中断有不同的跳转地址,实验中不开启,共用一个处理程序地址。
该地址是由指令设定(即写在exp\_test.s$\to$转换成rom.hex$\to$读到ROM\_D)。\textbf{不可以在Verilog中设定}(Verilog是硬件设计语言,
写在Verilog中等于是在把这一地址固定到了芯片中)。\\


\textbf{Q3:关于mstatus}

A:实验中要根据mstatus中的MIE域来响应中断,进入异常处理程序时要把MIE域存到MPIE域并将MIE域置零,mret退出时则恢复。
这一处理是因为外部硬件中断随时可能发生且时长不定,不关闭中断响应整个程序就会被卡死。\\


\textbf{Q4:关于mepc}

A:对于异常,mepc保存异常的指令地址并取消这一指令的写回;对于中断,mepc不取消当前要写回的指令并保存下一\textbf{待执行}指令地址。(思考:下一待执行指令一定是PC+4吗?)
这背后的逻辑是不重复执行同一条指令、不漏执行指令。实验中没有对异常做实际解决,因此在处理程序中手动将mepc+4以跳过异常指令,这一软件行为不影响
硬件逻辑。\\


\textbf{Q5:CSRRegs只有一个写口,要改多个怎么办?}

A:用多周期的状态机或是修改CSRRegs。用后一方法时要注意,上述几个寄存器同样要可以在使用csrrw指令进行读写时通过寄存器号来寻址。\\


\textbf{Q6:如何确定哪个阶段来处理mret、CSR读写?}

A:单考虑mret、CSR读写,则在译码后写回前任一阶段做都可以实现指令目的。但由于实验中是在WB阶段处理异常,且处理时要写部分CSR,
因而要考虑CSR的数据竞争问题,最早只能在MEM阶段做,此时前一条指令已经到达WB,ExceptionUnit可以按优先级来处理。\\


\textbf{Q7:仿真没有中断信号?}

A:请自己修改core\_sim.v,将interrupter的输入改为一个自定义变量reg,根据仿真找到一个比较空闲的时间点t,在initial块中添加改变信号的代码
“\#t XXX=1;”(意思是在上一行代码时间点之后过tns把XXX置为1),并自己观察相关信号调试。下板后的中断开关是SW[12](在top.v中传入)。


\end{document}