\documentclass{article}
\usepackage{ctex}
\usepackage{hyperref}
\usepackage[hmargin=1.25in, vmargin=1in]{geometry}

\usepackage{fontspec}
\setmainfont{等线}
\usepackage{xeCJK}
\setCJKmainfont{等线}


\title{\textbf{体系实验一FAQs}\vspace{-2em}}
\author{}
\date{}
\begin{document}

\maketitle

\textbf{前言:}
硬件设计有多种不同的方式可以达到相同的效果。在进行实验时,建议以需求驱动出发来设计,即对于要设计/补全的模块/信号,从整体框架上了解它的输出要起到什么
功能,然后根据这一功能需要哪些信号值再从相应模块引入,不建议单在一个模块内根据上下文、信号名语义来推断补全。鼓励根据自己的设计需求,在不改变整体框架逻辑
的情况下对参考代码各模块的输入输出和内部逻辑做修改。\\


\textbf{Q1:关于开发工具}

A:实验室板子是xc7k325tffg676-2L。Vivado可从官网\href{https://xilinx.com/support/download.html}{https://xilinx.com/support/download.html}
下载,版本不限。安装时设备勾选只需选了Kintex-7系列即可。免费版/标准版没有325这一型号的板子,可以进行仿真但无法生成bitstream,
需自行找许可证激活高级版,或用实验室台式机上的Vivado进行生成。\\


\textbf{Q2:仿真为什么全是红线?}

A:参考代码的ROM\_D.v(RAM\_B.v同理)中用了\$readmemh("rom.hex", inst\_data)这一相对路径,在综合生成中,这一路径是相对于ROM\_D.v,但在
仿真中,这一路径是相对于仿真程序的exe(位于XXXX.sim/sim\_1/behav/xsim),因此需要把rom.hex和ram.hex复制到这一目录中。\\


\textbf{Q3:关于Code2Inst}

A:完全注释掉Code2Inst后生成时会有“has undefined contents and is considered a black box”的报错,可以在里面添加下面的代码解决。
直接取消注释也可以,但会增加生成时间。

initial begin

\indent\indent inst="XXX";

end\\


\textbf{Q4:下板后如何操作}

A:在Program device前先把最右边的开关SW[0]打开(关闭是自动运行,打开是手动单步运行),程序下板后用阵列按键的左下角按键单步运行。
VGA屏幕上方区域是寄存器值,中间区域是由Test\_signal传递的各信号,下方区域目前未使用。\\


\textbf{Q5:hazard\_optype是什么意思?}

A:是用以区分ALU、访存等指令类别的信息。参考代码的Hazard\_Detection\_Unit输入只有hazard\_optype\_ID,没有别的阶段的该信号,有几种方式解决:
1、内部增加hazard\_optype\_EXE等reg并随时钟更新;2、在外部的阶段锁存器增加存储这一信号;3、改用其它信号。\\



\textbf{Q6:实验验收DDL?}

A:学在浙大上的工程提交时间之前的最后一次实验课为验收DDL。可延迟验收,以延迟的实验课次数来记penalty。



\end{document}